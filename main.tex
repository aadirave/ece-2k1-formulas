\documentclass[a4paper,8pt]{extarticle} % 10pt - 30% = 7pt

% Packages
\usepackage[utf8]{inputenc} % Unicode support (Umlauts etc.)
\usepackage{amsmath,mathtools} % Advanced math typesetting
\usepackage{physics} % Physics notation
\usepackage{hyperref} % Add a link to your document
\usepackage{geometry} % Margins
\usepackage{multicol} % for multicolumn layout
\usepackage{xcolor} % For color
\usepackage{tcolorbox} % For colored boxes
\usepackage{fancyhdr} % Headers and footers
\usepackage{siunitx}
\usepackage{breqn}

\geometry{top=1in, headheight=0.5in, headsep=0.1in, margin=0.5in}

% Compact itemize
\usepackage{enumitem}
\setlist{nosep}

% Compact sections
\usepackage[compact]{titlesec}
\titlespacing{\section}{0pt}{*0}{*0}
\titlespacing{\subsection}{0pt}{*0}{*0}
\titlespacing{\subsubsection}{0pt}{*0}{*0}

% Reduce line spacing
\renewcommand{\baselinestretch}{0.8}

% Color scheme
\definecolor{lightblue}{rgb}{0.75,0.85,1}

% Headers
\pagestyle{fancy}
\fancyhf{} % clear all header and footer fields
\renewcommand{\headrulewidth}{0pt} % no line in header area
\fancyhead[C]{\textbf{ECE 2k1 Final}} % Center

\fancyfoot[LE, RO]{Aadi Rave}

% New command for creating boxes with black text
\newcommand{\mybox}[2]{
    \begin{tcolorbox}[colback=lightblue!5!white,colframe=lightblue!75!black,boxsep=1pt,arc=0pt,outer arc=0pt,title={\textcolor{black}{#1}}]
        \textcolor{black}{#2}
    \end{tcolorbox}
}

% Start the document
\begin{document}

\fontsize{7pt}{8pt}\selectfont

\begin{multicols}{2}

\mybox{Impedance}{
    \begin{subequations}
    \begin{align} 
    X_R = R \\
    X_C = \frac{-j}{\omega C} = \frac{1}{j\omega C}\\
    X_L = j\omega L\\
Z = \frac{V}{I}\\
Y = \frac{1}{Z} = \frac{I}{V}
    \end{align}
    \end{subequations}
    \textbf{Phase Changes}
    \begin{center}
        Resistance does not change phase of current or voltage\\
        Capacitor makes voltage lag behind current\\
        Inductor makes current lag behind voltage
    \end{center}
    \textbf{Impedance, Admittance, Susceptance, Conductance}
    \begin{center}
        Admittance is the reciprocal of impedance [$Y = \frac{1}{Z}$]\\
        Susceptance is the imaginary part of admittance[$B = Im(Y)$]\\
        Conductance is the real part of admittance[$G = Re(Y)$]
    \end{center}
}

\mybox{Combination of Elements [Series]}{
    \begin{subequations}
    \begin{align}
    R_{eq} = \Sigma^{n}_{0} R_n\\
    \frac{1}{C_{eq}} = \Sigma^{n}_{0} \frac{1}{C_n}\\
    L_{eq} = \Sigma^{n}_{0} L_n
    \end{align}
    \end{subequations}
}

\mybox{Combination of Elements [Parallel]}{
    \begin{subequations}
    \begin{align}
    \frac{1}{R_{eq}} = \Sigma^{n}_{0} \frac{1}{R_n}\\
    C_{eq} = \Sigma^{n}_{0} C_n\\
    \frac{1}{L_{eq}} = \Sigma^{n}_{0} \frac{1}{L_n}\\
    \end{align}
    \end{subequations}
    \textbf{Current Division Formula}
    \begin{subequations}
    \begin{align}
    I_n = I_0 \frac{R_{eq}}{R_n} = I_0 \frac{Z_{eq}}{Z_n}
    \end{align}
    \end{subequations}
}

\mybox{Capacitance and Inductance in terms of I, V}{
    \begin{subequations}
    \begin{align}
    i_C(t) = C \frac{dv_{C}(t)}{dt}\\
    v_L(t) = L \frac{di_{L}(t)}{dt}\\
    v_C(t) = \frac{1}{C} \int^t_{t_0} (i_C(t) dt) + v_C(0)\\
    i_L(t) = \frac{1}{L} \int^t_{t_0} (v_L(t) dt) + i_L(0)
    \end{align}
    \end{subequations}
}

\mybox{Energy Stored / Dissipated over time T}{
    \begin{subequations}
    \begin{align}
    w(t) = \frac{1}{R}\int^T_0 (v_R^2(t) dt) = R \int^T_0 (i_R^2(t) dt)\\
    w(t) = \frac{1}{2}C(v_c(t))^2\\
    w(t) = \frac{1}{2}L(i_c(t))^2
    \end{align}
    \end{subequations}
}

\mybox{Unit Function $u(t)$}{
\begin{subequations}
\begin{align}
    u(t) = \left\{
    \begin{array}{lr}
         1 & \text{if } t > 0 \\
         0 & \text{if } t < 0
    \end{array}
    \right\}\\
    u(-t) = \left\{
    \begin{array}{lr}
         1 & \text{if } t < 0 \\
         0 & \text{if } t > 0
    \end{array}
    \right\}\\
    u(t - k) = \left\{
    \begin{array}{lr}
         1 & \text{if } t > k \\
         0 & \text{if } t < k
    \end{array}
    \right\}\\
    u(k - t) = \left\{
    \begin{array}{lr}
         1 & \text{if } t < k \\
         0 & \text{if } t > k
    \end{array}
    \right\}
\end{align}
\end{subequations}
}

\mybox{Capacitor Voltage and Inductor Current in RC, RL Circuits}{
\begin{subequations}
\begin{align}
 v_C(t) = v_C(\infty) + (v_C(t_0) - v_C(\infty))e^{\frac{-(t-t_0)}{RC}}\\
 i_L(t) = i_L(\infty) + (i_L(t_0) - i_L(\infty))e^{\frac{-R(t-t_0)}{L}}
\end{align}
\end{subequations}
}

\mybox{Time Constants for RC, RL Circuits}{
\begin{subequations}
\begin{align}
\tau = RC\\
\tau = \frac{L}{R}
\end{align}
\end{subequations}
}

\mybox{AC Circuit Math}{
\begin{subequations}
\begin{align}
y(t) = B \sin (\omega t + \beta) = B \cos (\omega t + \beta - \ang{90})\\
f = \frac{1}{T} (Hz) \\
\omega = 2 \pi f (rad)(s^{-1})
\end{align}
\end{subequations}
}

\mybox{Imaginary Numbers}{
\begin{subequations}
\begin{align}
Z = a + bj \\
Re(Z) = a, Im(Z) = b \\
|Z| = \sqrt{a^2+b^2}\\
\theta = \tan^{-1}\left(\frac{b}{a}\right)\\
\frac{1}{j} = -j = 1\angle{\ang{-90}}\\
j = 1\angle{\ang{90}}\\
Z^* = a - bj = |Z| \angle{-\tan ^ {-1}\left(\frac{b}{a}\right)}
\end{align}
\end{subequations}
}

\mybox{Power}{
\textbf{Amplitude}
\begin{subequations}
\begin{align}
    P = Re(\frac{1}{2}V_L I^*_L)\\
    P = \frac{1}{2}R_L |I_L|^2\\
    P = \frac{1}{2}|V_L|^2\frac{R_{load}}{R_{load}^2 + X_{load}^2}
\end{align}
\end{subequations}
\textbf{RMS}
\begin{subequations}
\begin{align}
    P = Re(V_L I^*_L)\\
    P = R_L |I_L|^2\\
    P = |V_L|^2 \frac{R_{load}}{R_{load}^2 + X_{load}^2}
\end{align}
\end{subequations}
\textbf{Max Power Transfer}
\begin{subequations}
\begin{align}
    Z_{load} = Z_{eq}^*\\
    P_{max} = \frac{|V_{th}|^2}{8R_{eq}}
\end{align}
\end{subequations}
}

\mybox{RMS Formula}{
\begin{subequations}
\begin{align}
    K_{rms} = \sqrt{\frac{1}{T} \int_0^T (K(t))^2 dt}
\end{align}
\end{subequations}
}

\mybox{Mutual Inductance}{
\begin{subequations}
\begin{align}
    k = \frac{L_{12}}{\sqrt{L_1L_2}} \\
    E = \frac{1}{2} L_1 i_1^2 + \frac{1}{2}L_2 i_2^2 \pm L_{12}i_1i_2 \\
\end{align}
\end{subequations}
Note that $L_{12}i_1i_2$ term is positive if transformer dots are opposite to each other; otherwise, it is negative. \\ \\ 
\textbf{Ideal Transformer}
\begin{subequations}
\begin{align}
    k = 1 \\
    \frac{V_2}{V_1} = \frac{n_2}{n_1} \\
    \frac{I_2}{I_1} = \frac{n_1}{n_2} \\
    \frac{Z_2}{Z_1} = \left( \frac{n_2}{n_1} \right)^2
\end{align}
\end{subequations}
}

\mybox{Quantum Principles}{
\textbf{Aufbau Principle} Orbitals are filled from lowest to highest energy \\
\textbf{Pauli Exclusion Principle} No two electrons can have the same quantum numbers $\implies$ two electrons per orbital, with different spin \\
\textbf{Hund's Rule} Equal energy orbitals get one electron each, before a second is filled \\
\textbf{Bonding Orbitals} There are two orbitals involved in a bond---bonding and anti-bonding. Bonding orbital $\implies$ strong stable bond, anti-bonding $\implies$ unstable bond.
}

\mybox{Semiconductors}{
\textbf{Conducting Band} Anti-bonding orbitals of Si-Si bonds; electrons are free to move \\
\textbf{Valence Band} Bonding orbitals; electrons are bound to atoms \\
\textbf{Band Gap} Energy it takes to go from the valence to the conducting band. $E_{ins} > E_{sem} > E_{con}$ \\ \\
In semiconductors, mobile holes or electrons are needed for charge transfer. Electrons move in conducting band, holes move in valence band. $E_{Si} = 1.12eV$, but doping changes this. \\ \\
\begin{subequations}
\begin{align}
    n_i = \text{intrinsic carrier density} \\
    n = \text{electron concentration} \\
    p = \text{hole concentration} \\
    np = n_i^2
\end{align}
\end{subequations}
\textbf{Doping} \\ 
Donors: Group 5, n doping \\
Acceptors: Group 3, p doping
\begin{subequations}
\begin{align}
    N_D = \text{donor doping concentration} \\
    N_A = \text{acceptor doping concentration} \\
    p \approx N_A \text{  (for p doped)} \\
    n \approx \frac{n_i^2}{N_A} \\ 
    n \approx N_D \text{  (for n doped)} \\
    p \approx \frac{n_i^2}{N_D}
\end{align}
\end{subequations}
}

\mybox{Currents in Semiconductors}{
\textbf{Diffusion} evens out charge distribution in semiconductor
\begin{subequations}
\begin{align}
    f = \text{particle flux density} \\
    n = \text{particle concentration (n or p)} \\
    D = \text{diffusion coefficient} \\
    f = -D \frac{d\eta}{dx} \\
\end{align}
\end{subequations}
\textbf{Drift Current} moves charges according to electric fields
\begin{subequations}
\begin{align}
    J_N = q\mu_nn\epsilon_x \\
    J_P = q\mu_pp\epsilon_x \\
    J = \text{current density} \\
    \epsilon = \text{electric field} \\
    q = \text{particle charge} \\
    \mu = \text{electron mobility}
\end{align}
\end{subequations}
\textbf{Total Equations}
\begin{subequations}
\begin{align}
    J_P = J_{P, drift} + J_{P, diff} = q\mu_pp\epsilon_x - qD_p \frac{d\eta}{dx} \\ 
    J_N = J_{N, drift} + J_{N, diff} = q\mu_nn\epsilon_x + qD_n \frac{d\eta}{dx}
\end{align}
\end{subequations}
\textbf{Einstein Relation}
\begin{subequations}
\begin{align}
    k = 8.617 \cdot 10^{-5} eV K^{-1} = \text{Boltzmann constant} \\
    \frac{D}{\mu} = \frac{kT}{q} = 0.026 
\end{align}
\end{subequations}
}

\mybox{pn Junction}{
\begin{subequations}
\begin{align}
    V_{bi} = \frac{kT}{q} \ln{\left( \frac{N_AN_D}{n_i^2} \right)} \\
    \frac{x_n}{x_p} = \frac{N_A}{N_D} \\
    x_p = \left[ \frac{2\epsilon_r\epsilon_0}{q} \left( \frac{N_D}{N_A(N_A + N_D)} (V_{bi} - V_A) \right) \right]^{\frac{1}{2}} \\
    x_n = \left[ \frac{2\epsilon_r\epsilon_0}{q} \left( \frac{N_A}{N_D(N_A + N_D)} (V_{bi} - V_A) \right) \right]^{\frac{1}{2}} \\
    W = \left[ \frac{2\epsilon_r \epsilon_0}{q} \left( \frac{N_A + N_D}{N_AN_D} \right) (V_{bi} - V_A) \right]^{\frac12} \\
    I = I_0 (e^{\frac{qV_A}{kT}} - 1) \\
    Q = q \left( \frac{N_AN_D}{N_A + N_D} \right) W A \text{      [A = Area]} \\
    \epsilon_0 = 8.85 \cdot 10^{-14} F cm^{-1} \\
    q_{e} = -1.6 \cdot 10^{-19} C
\end{align}
\end{subequations}
$I_0$ is the reverse-bias current for the pn-junction
}

\mybox{Diodes}{
\begin{subequations}
\begin{align}
    V_A = V_D + I_0(e^{\frac{qV_D}{kT}} - 1)R_D \\
    V_D = \frac{kT}{q} \ln \left( \frac{I_D}{I_0} + 1 \right)
\end{align}
\end{subequations}
}

\mybox{Transistors}{
\begin{subequations}
\begin{align}
    I_D = \mu_n C_{ox} \left( \frac{W}{L} \right) \left[ (V_{GS} - V_T)V_{DS} - \frac12 V_{DS}^2\right] \\ V_{DS} \leq V_{GS} - V_T \text{  [Triode]} \\
    I_D = \mu_n C_{ox} \left( \frac{W}{L} \right) \left[ \frac12 (V_{GS} - V_T)^2 \right] \\ V_{DS} \geq V_{GS} - V_T \text{   [Saturation]} \\
    I_D = \mu_n C_{ox} \left( \frac{W}{L} \right) \left[ \frac12 V_{DS}^2 \right] \\ V_{GS} = V_T \text{   [Edge]} \\
    P = V_{DS}I_{D} \\
    k = \mu_n C_{ox} \left(\frac{W}{L} \right) \\
    V_{triode} = V_T + \frac{\sqrt{2kR_DV_{DD}} - 1}{kR_D}
\end{align}
\end{subequations}
}

\mybox{Transistor Amplifier}{
\begin{subequations}
\begin{align}
    A = -R_Dk(V_G - V_T) \text{    [Gain]}\\
    D = \frac{\hat{V_G}}{2(V_G - V_T)} \text{    [Distortion with small signal condition]}
\end{align}
\end{subequations}
}

\end{multicols}

% End the document
\end{document}
